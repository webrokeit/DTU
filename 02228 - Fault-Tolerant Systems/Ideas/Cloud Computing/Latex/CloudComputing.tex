\documentclass[11pt]{report}
\usepackage{geometry}
\usepackage{graphicx}
\usepackage{titling}
\usepackage{fancyhdr}
\usepackage{enumerate}
\usepackage{latexsym}
\usepackage{float}
\usepackage[labelfont=bf]{caption}
\usepackage[usenames,dvipsnames,svgnames,table]{xcolor}
\usepackage{listings}
\usepackage{hyperref}
\parindent=0pt
%\frenchspacing

\pagestyle{fancy}

\fancyhead[L]{\slshape\footnotesize December 9, 2014\\\textsc{02228 Fault-Tolerant Systems}}
\fancyhead[R]{\slshape\footnotesize \textsc{Andreas Kjeldsen (s092638)}\\\textsc{Morten Eskesen (s133304)}}
\fancyfoot[C]{\thepage}
\newcommand{\HRule}{\rule{\linewidth}{0.5mm}}

\begin{document}

\begin{titlepage}
\begin{center}

\includegraphics[scale=2.0]{../GFX/dtu_logo.pdf}\\[1cm]
\textsc{\LARGE Technical University of Denmark}\\[1.5cm]
\textsc{\Large 02228 Fault-Tolerant Systems}\\[0.5cm]

% Title
\HRule \\[0.4cm]
{\huge \bfseries Fault-Tolerant Cloud Computing Architectures}\\[0.1cm]
\HRule \\[1.5cm]

% Author and supervisor
{\large
\emph{Authors:} \\[10pt]
Andreas Hallberg \textsc{Kjeldsen}\\
\emph{s092638@student.dtu.dk} \\[10pt]
Morten Chabert \textsc{Eskesen}\\
\emph{s133304@student.dtu.dk}
}
\vfill

% Bottom of the page
{\large December 9, 2014}

\end{center}
\end{titlepage}

\chapter{Introduction}
In this report we will describe what cloud computing is, further we will give a detailed description of the architecture and fault-tolerant features of two cloud system, at last we will compare how the systems handle failures and discuss the pros and cons of these methods. As a result of the comparison, we will be able to conclude on what the systems do well and where they might be able to improve.

\section{Scope}
We will focus on the fault-tolerant features of the cloud computing architecture within the two selected cloud computing systems. We have chosen to focus on Amazon Web Services and Google Cloud Platform. We have chosen these cloud computing systems because both systems are among the most popular\footnote{Popular means that they are among the most commonly used platforms for enterprise cloud developers} cloud computing systems \cite{cloudsurvey}.

\section{Cloud Computing}
The National Institute of Standards and Technology is a federal technology agency in the United States of America. They define cloud computing by the following:
\begin{center}
\emph{"Cloud computing is a model for enabling ubiquitous, convenient, on-demand network access to a shared pool of configurable computing resources that can be rapidly provisioned and released with minimal management effort or service provider interaction."} \cite{clouddefinition}
\end{center}
This definition states that shared networks, servers, applications, services etc can easily be distributed globally and quickly maintained by using cloud computing.\\

There are five essential characteristics of the cloud computing model
\begin{description}
\item[On-demand self-service] No required human interaction when needing more or less computing capabilities
\item[Broad network access] Capabilities are accessed through standard mechanisms and available over the network
\item[Resource pooling] Computing resources are pooled in order to serve multiple consumers 
\item[Rapid elasticity] Capabilities can be elastically released to scale rapidly according to the demand
\item[Measured service] In an automatic way the cloud computing systems control and optimize recourse use
\end{description}

There are four different deployment models. One cloud infrastructure is for exclusive use by a single organization comprising the multiple consumers - the \emph{private cloud}. Another cloud infrastructure is for exclusive use by a specific community of users from organizations with shared concerns - the \emph{community cloud}. The \emph{public cloud} is an infrastructure open for use by the general public. The last cloud infrastructure is a mixture of two or more distinct cloud infrastructures that remain unique entities - the \emph{hybrid cloud}.

\section{Fault tolerance in cloud computing}
Fault tolerance is a key factor for cloud computing systems due to the rapid exponential growth in use of cloud computing \cite{faulttolerancetechniques}. The purpose of fault tolerance in any system is to achieve robustness and dependability. Fault tolerance policies and techniques allow us to classify this techniques into 2 types
\begin{description}
\item[Proactive fault tolerance policy] aims to avoid recovering from fault, errors and failure by predicting them and replacing the suspicious component. This means detecting problems before they actually occur.
\item[Reactive fault tolerance policy] reduces the effect of failures when the failure actually occurs.
\end{description}
These policies can be divided into two further sub techniques error processing and fault treatment. The aim of error processing is to remove errors from the computational state and the aim of fault treatment is to prevent faults from reoccurring.

\chapter{Amazon Web Services}
The Amazon Web Services, henceforth \emph{AWS}, is a collection of remote computing services which make a cloud computing platform launched by Amazon.com in 2006. This chapter will outline the architecture of the cloud computing platform and describe the fault-tolerant features the platform has.

\section{Architecture}
\section{Fault-Tolerant Features}

\chapter{Google Cloud Platform}
\section{Architecture}
\section{Fault-Tolerant Features}

\chapter{Comparison of Failure Handling}
List of faults that the systems handle along with a description of how it's handled and why it works. If the methods for handling the failure differ, we will discuss the methods, highlighting their pros and cons.

\chapter{Conclusion}
Conclude on our findings, focus on what the systems do well and where it might be possible to improve.

\begin{thebibliography}{99}

\bibitem{AWS_ac_ra_ftha_04}
	Amazon Web Services Reference Architectures, \emph{Fault Tolerance \& High Availability}, 2014.
	\url{http://media.amazonwebservices.com/architecturecenter/AWS_ac_ra_ftha_04.pdf}
	
\bibitem{AWS_building_fault_tolerant_applications}
	Amazon Web Services Whitepapers, \emph{Building Fault Tolerant Applications}, October 2011.
	\url{http://media.amazonwebservices.com/AWS_Building_Fault_Tolerant_Applications.pdf}
	
\bibitem{AWS_slideshare_designing_fault_tolerant_applications}
	Amazon Web Services, \emph{Designing Fault-Tolerant Applications}, Slides, July 2011.
	\url{http://www.slideshare.net/AmazonWebServices/base-camp-awsdesigningfaulttolerantapplications}
	
\bibitem{AWS_youtube_designing_fault_tolerant_applications}
	Amazon Web Services, \emph{Designing Fault-Tolerant Applications}, YouTube, July 2011.
	\url{https://www.youtube.com/watch?v=9BrmHoyFJUY}
	
\bibitem{google_cloud_platform_documentation}
	Google, \emph{Google Cloud Platform Documentation}.
	\url{https://cloud.google.com/docs/}	
	
\bibitem{google_dataplace_for_fault_tolerance}
	Google Patents, \emph{Data placement for fault tolerance}, February 2006.
	\url{http://www.google.com/patents/US7000141}
	
\bibitem{google_app_engine_scalability_2009}
	Google I/O, \emph{App Engine: Scalability, Fault Tolerance, and Integrating Amazon EC2}, YouTube, June 2006.
	\url{https://www.youtube.com/watch?v=p4F62q1kJ7I}
	
\bibitem{google_cloud_platform_blog}
	Google, \emph{Google Cloud Platform Blog}.
	\url{http://googlecloudplatform.blogspot.dk/}
	
\bibitem{google_cloud_platform_gets_developer_enhancements}
	Todd R. Weiss, \emph{Google Cloud Platform Gets Developer Enhancements}, August 2013.
	\url{http://www.eweek.com/cloud/google-cloud-platform-gets-developer-enhancements}
	
\bibitem{cloudsurvey}
	Larry Dignan, \emph{Amazon Web Services, Windows Azure top cloud dev choices, says survey}, August 2013.
	\href{http://www.zdnet.com/amazon-web-services-windows-azure-top-cloud-dev-choices-says-survey-7000019115/}
	{Online article}
	%% Linket er for stort til margenen somehow, quick fix lige pt.. :)
	
\bibitem{clouddefinition}
	National Institute of Standards and Technology, \emph{The NIST Definition of Cloud Computing}, October 2011.
	\url{http://csrc.nist.gov/publications/nistpubs/800-145/SP800-145.pdf}
	
\bibitem{faulttolerancetechniques}
	Prasenjit Kumar Patra, Harshpreet Singh and Gurpreet Singh, \emph{Fault Tolerance Techniques and Comparative Implementation in Cloud Computing}, February 2013.
	International Journal of Computer Applications.

\end{thebibliography}

\end{document}
