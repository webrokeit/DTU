\documentclass[11pt]{report}
\usepackage{geometry}
\usepackage{graphicx}
\usepackage{titling}
\usepackage{fancyhdr}
\usepackage{enumerate}
\usepackage{latexsym}
\usepackage{float}
\usepackage[labelfont=bf]{caption}
\usepackage[usenames,dvipsnames,svgnames,table]{xcolor}
\usepackage{listings}
\usepackage{hyperref}
\parindent=0pt
%\frenchspacing

\pagestyle{fancy}

\fancyhead[L]{\slshape\footnotesize October 7, 2014\\\textsc{02228 Fault-Tolerant Systems}}
\fancyhead[R]{\slshape\footnotesize \textsc{Andreas Kjeldsen (s092638)}\\\textsc{Morten Eskesen (s133304)}}
\fancyfoot[C]{\thepage}
\newcommand{\HRule}{\rule{\linewidth}{0.5mm}}

\begin{document}

\begin{titlepage}
\begin{center}

\includegraphics[scale=2.0]{../GFX/dtu_logo.pdf}\\[1cm]
\textsc{\LARGE Technical University of Denmark}\\[1.5cm]
\textsc{\Large 02228 Fault-Tolerant Systems}\\[0.5cm]

% Title
\HRule \\[0.4cm]
{\huge \bfseries Fault-Tolerant Cloud Computing Architectures}\\[0.1cm]
\HRule \\[1.5cm]

% Author and supervisor
{\large
\emph{Authors:} \\[10pt]
Andreas Hallberg \textsc{Kjeldsen}\\
\emph{s092638@student.dtu.dk} \\[10pt]
Morten Chabert \textsc{Eskesen}\\
\emph{s133304@student.dtu.dk}
}
\vfill

% Bottom of the page
{\large October 7, 2014}

\end{center}
\end{titlepage}

\chapter{Introduction}
In this report we will describe what cloud computing is, further we will give a detailed description of the architecture and fault-tolerant features of two cloud system, at last we will compare how the systems handle failures and discuss the pros and cons of these methods. As a result of the comparison, we will be able to conclude on what the systems do well and where they might be able to improve.

\section{Scope}
We will focus on the fault-tolerant features of the cloud computing architecture within the two selected cloud systems (Amazon Web Services and Google Cluster).

\section{Cloud Computing}
Brief description of what cloud computing is...

\chapter{Amazon Web Services}
The Amazon Web Services, henceforth \emph{AWS}....

\section{Architecture}
\section{Fault-Tolerant Features}

\chapter{Google Cluster}
\section{Architecture}
\section{Fault-Tolerant Features}

\chapter{Comparison of Failure Handling}
List of faults that the systems handle along with a description of how it's handled and why it works. If the methods for handling the failure differ, we will discuss the methods, highlighting their pros and cons.

\chapter{Conclusion}
Conclude on our findings, focus on what the systems do well and where it might be possible to improve.

\begin{thebibliography}{99}

\bibitem{AWS_ac_ra_ftha_04}
	AWS Reference Architectures, \emph{Fault Tolerance \& High Availability}, 2014.
	\url{http://media.amazonwebservices.com/architecturecenter/AWS_ac_ra_ftha_04.pdf}

\end{thebibliography}

\end{document}
