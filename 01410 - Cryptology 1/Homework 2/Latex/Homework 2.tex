\documentclass[11pt]{report}
\usepackage{geometry}
\usepackage{graphicx}
\usepackage{titling}
\usepackage{fancyhdr}
\usepackage{enumerate}
\usepackage{ulem}
\usepackage{latexsym}
\usepackage{float}
\usepackage{amssymb}
\usepackage{amsthm}
\usepackage{amsfonts}
\usepackage{amsmath}
\usepackage[labelfont=bf]{caption}
\usepackage[usenames,dvipsnames,svgnames,table]{xcolor}
\usepackage{listings}
\usepackage{lmodern}
\usepackage{../Maple/maplestd2e}
\parindent=0pt
%\frenchspacing

\pagestyle{fancy}

\fancyhead[L]{\slshape\footnotesize April 7, 2014\\\textsc{01410 Cryptology 1}}
\fancyhead[R]{\slshape\footnotesize \textsc{Andreas Kjeldsen (s092638)}\\\textsc{Morten Eskesen (s133304)}}
\fancyfoot[C]{\thepage}
\newcommand{\HRule}{\rule{\linewidth}{0.5mm}}

\begin{document}

\begin{titlepage}
\begin{center}

\includegraphics[scale=2.0]{../GFX/dtu_logo.pdf}\\[1cm]
\textsc{\LARGE Technical University of Denmark}\\[1.5cm]
\textsc{\Large 01410 Cryptology 1}\\[0.5cm]

% Title
\HRule \\[0.4cm]
{\huge \bfseries Homework 2}\\[0.1cm]
\HRule \\[1.5cm]

% Author and supervisor
{\large
\emph{Authors:} \\[10pt]
Andreas Hallberg \textsc{Kjeldsen}\\
\emph{s092638@student.dtu.dk} \\[10pt]
Morten Chabert \textsc{Eskesen}\\
\emph{s133304@student.dtu.dk}
}
\vfill

% Bottom of the page
{\large April 7, 2014}

\end{center}
\end{titlepage}

\section*{Exercise 2.1}
\subsection*{2.1.1}
We have to show that $m^{e\tilde{d}} \equiv m$ mod $n$ for all $m \in \mathbb{Z}_{n}$.\\
The keys $e$ and $\tilde{d}$ are chosen such that.
\begin{center}
$e\tilde{d} \equiv 1$ mod $\frac{(p-1)(q-1)}{gcd(p-1,q-1)}$
\end{center}
This means that for some positive integer $k$
$$e\tilde{d} = 1 + k\frac{(p-1)(q-1)}{gcd(p-1,q-1)}$$
We can rewrite this expression and get
\begin{center}
$m^{e\tilde{d}}$ mod $n$ = $m^{1+k\frac{(p-1)(q-1)}{gcd(p-1,q-1)}}$ mod $n$
\end{center}
for some integer $k$.\\
If two integers $x$ and $y$ are congruent modulo $n$ then they are also congruent modulo $p$ and modulo $q$ because both $p$ and $q$ divide $n$. The Chinese Remainder Theorem tells us that the reverse implication is also true. This means that if $x$ and $y$ are congruent modulo $p$ and congruent modulo $q$, then they are also congruent modulo $n$.\\
We want to show that $m^{e\tilde{d}} \equiv m$ mod $n$ so if it will be sufficient to show that:
\begin{center}
$m^{e\tilde{d}} \equiv m$ mod $p$ and  $m^{e\tilde{d}} \equiv m$ mod $q$
\end{center}
First we will show that $m^{e\tilde{d}} \equiv m$ mod $p$. We have therefore have two cases to consider:
\begin{enumerate}
\item $p$ divides m
\item $p$ does not divide m.
\end{enumerate}
Case 1: If $p$ divides $m$, then $m \equiv 0$ mod $p$, but also $m^{e\tilde{d}} \equiv 0$ mod $p$, therefore $m^{e\tilde{d}} \equiv m$ mod $p$.\\
Case 2: If $p$ does not divide m then $m \in \mathbb{Z}^*_p$. By Fermat's Little Theorem we have $m^{p-1} \equiv 1$ mod $p$. Since $e\tilde{d} \equiv 1$ mod $\psi(n)$, we have that $psi(n)$ divides $e\tilde{d} - 1$. This equivalent with some integer $k$: $k \psi(n) = e\tilde{d} - 1$, so $e\tilde{d} = k \psi(n) + 1$ for some integer $k$. We therefore have:
\begin{center}
$m^{e\tilde{d}} = m^{k\psi(n)+1} = m * m^{k\frac{(p-1)(q-1)}{gcd(p-1,q-1)}}$ \\
$m^{e\tilde{d}} = m * (m^{p-1})^{k\frac{(q-1)}{gcd(p-1,q-1)}}$ \\
$m^{e\tilde{d}} \equiv m * 1^{k\frac{(q-1)}{gcd(p-1,q-1)}}$ mod $p$ \\
$m^{e\tilde{d}} \equiv m$ mod $p$.
\end{center}
We can do similar calculations to show that $m^{e\tilde{d}} \equiv m$ mod $q$ by replacing $p$ by $q$.\\
Therefore we have now shown for all $m \in \mathbb{Z}_n$ that
\begin{center}
$m^{e\tilde{d}} \equiv m$ mod $p$ and  $m^{e\tilde{d}} \equiv m$ mod $q$
\end{center}
Hence we can concluce that $m^{e\tilde{d}} \equiv m$ mod $n$ for all $m \in \mathbb{Z}_{n}$.
\subsection*{2.1.2}
Let $p = 881$, $q = 461$, and let $n = pq = 405141$. We have to show that $e = 3$ is an allowed encryption exponent for an RSA encryption system with modulus $n$. By the definition of RSA $e$ must be chosen such that $e$ and $\phi(n)$ are co-prime. Formally this means that $e \in \mathbb{Z}^*_{\phi(n)}$, where $\phi(n) = (p-1)(q-1)$.
$$gcd(3, (881-1)(461-1) = 1$$
This means that $e$ and $\phi(n)$ are co-prime and therefore $e=3$ is an allowed encryption exponent.

\subsection*{2.1.3}
We have to find $d_1$ such that $ed_1 \equiv 1$ mod $\phi(n)$.
\begin{mapleinput}
p := 881; q := 461; e := 3; \\
d1 := mod($e^{-1}$, (p-1)*(q-1))
\end{mapleinput}
Using the maple code above we find that $d_1 = 269867$
\subsection*{2.1.4}
We have to find $d_2$ such that $ed_2 \equiv 1$ mod $\psi(n)$
\begin{mapleinput}
p := 881; q := 461; e := 3; \\
d2 := mod($e^{-1}$, (p-1)*(q-1)/gcd(p-1, q-1))
\end{mapleinput}
Using the maple code above we find that $d_2 = 6747$.
\subsection*{2.1.5}
Choosing $\psi(n)$ instead of $\phi(n)$ in the congruence for d means that the decryption becomes faster since
$$lcm(p-1,q-1)=\frac{(p-1)(q-1)}{gcd(p-1,q-1)} \leq \frac{(p-1)(q-1)}{2}$$
Because $p$ and $q$ are odd primes $gcd(p-1,q-1) \geq 2$. 

\section*{Exercise 2.2}
\subsection*{2.2.a}
We have implemented trial division in maple with the following code:
\begin{mapleinput}
TrialDivision := proc (n::integer) \\
local i; \\
if n $\leq$ 1 then false \\
elif n = 2 then true \\
elif type(n, 'even') then false \\
else for i from 3 by 2 while i*i $\leq$ n do \\
if irem(n, i) = 0 then return false end if \\
end do; \\
true end if \\
end proc:
\end{mapleinput}

\begin{mapleinput}
result := 0; \\
for n from 25 to 25000 do \\
if TrialDivision(n) then result := result+1 end if \\
end do; \\
result;
\end{mapleinput}
Using this code we find that the number of primes $s$ between 25 and 25000 is 2753.
$$s = 2753$$

\subsection*{2.2.b}
We have implemented the Miller-Rabin algorithm with $k$ iterations in maple with the following code:
\begin{mapleinput}
MillerRabin := proc (n::integer, k::integer) \\
local x, r, roll, s, d, i, a; \\
s := n-1; d := 0; \\
while mod(s, 2) = 0 do \\
s := (1/2)*s; d := d+1 \\
end do; \\
for i to k do \\
roll := rand(2 .. n-1); \\
a := roll(); x := mod($a^s$, n); \\
if x = 1 or x = n-1 then next end if; \\
for r to d-1 do x := mod($x^2$, n); \\
if x = 1 then return false end if; \\
if x = n-1 then break end if \\
end do; \\
if x $\ne$ n-1 then return false end if \\
end do; \\
return true \\
end proc:
\end{mapleinput}
\subsection*{2.2.c}
We use this maple code below and define $k = 1,2,3,\dots$ to find the smallest number of iterations needed such that we gets the correct answer $s$.
\begin{mapleinput}
result := 0; \\
for n from 25 to 25000 do \\
if MillerRabin2(n, k) then result := result+1 end if \\
end do; \\
result;
\end{mapleinput}
This gives us the following table:
\begin{center}
\begin{tabular}{| c | c | c | c | c |}
\hline
k & 1 & 2 & 3 & 4 \\ \hline
s & 2792 & 2755 & 2754 & 2753 \\ \hline
\end{tabular}
\end{center}
With $k = 4$ iterations we get the correct answer for $s$ which is 2753.
\end{document}
