\documentclass[11pt]{report}
\usepackage{geometry}
\usepackage{graphicx}
\usepackage{titling}
\usepackage{fancyhdr}
\usepackage{enumerate}
\usepackage{ulem}
\usepackage{latexsym}
\usepackage{amssymb}
\usepackage{amsthm}
\usepackage{amsfonts}
\usepackage{amsmath}
\usepackage[labelfont=bf]{caption}
\usepackage[usenames,dvipsnames,svgnames,table]{xcolor}
\usepackage{listings}
\usepackage{lmodern}
\parindent=0pt
%\frenchspacing

\pagestyle{fancy}

\fancyhead[L]{\slshape\footnotesize March 3, 2014\\\textsc{01410 Cryptology 1}}
\fancyhead[R]{\slshape\footnotesize \textsc{Andreas Kjeldsen (s092638)}\\\textsc{Morten Eskesen (s133304)}}
\fancyfoot[C]{\thepage}
\newcommand{\HRule}{\rule{\linewidth}{0.5mm}}

\begin{document}

\begin{titlepage}
\begin{center}

\includegraphics[scale=2.0]{../GFX/dtu_logo.pdf}\\[1cm]
\textsc{\LARGE Technical University of Denmark}\\[1.5cm]
\textsc{\Large 01410 Cryptology 1}\\[0.5cm]

% Title
\HRule \\[0.4cm]
{\huge \bfseries Homework 1}\\[0.1cm]
\HRule \\[1.5cm]

% Author and supervisor
{\large
\emph{Authors:} \\[10pt]
Andreas Hallberg \textsc{Kjeldsen}\\
\emph{s092638@student.dtu.dk} \\[10pt]
Morten Chabert \textsc{Eskesen}\\
\emph{s133304@student.dtu.dk}
}
\vfill

% Bottom of the page
{\large March 3, 2014}

\end{center}
\end{titlepage}

\section*{Exercise 1.1}
\subsection*{1.1.1}
We're considering the Hill cipher. We have plaintext $\mathcal{P}$, {\tt\bfseries crypto}, which map to values {\tt\bfseries [2,17,24,15,19,14]}. We also have ciphertext $\mathcal{C}$, {\tt\bfseries LSDKDH}, which map to values {\tt\bfseries [11,18,3,10,3,7]}. We have that $\mathcal{P} = \mathcal{C} = \mathbb{Z}_{26}$.
\\
\\
We wish to find the $2 \times 2$ key matrix $K$ used in the encryption. To do this we first split the plaintext and ciphertext into pairs of two, this gives that {\tt\bfseries cr} is encrypted to {\tt\bfseries LS}, {\tt\bfseries yp} to {\tt\bfseries DK} and {\tt\bfseries to} to {\tt\bfseries DH}. Now we must find two pairs we can use to figure out the key. We start off with {\tt\bfseries cr} and {\tt\bfseries yp}.\\
\\
First we wish to check if the plaintext message $m$ is invertible.
$$m = \begin{bmatrix}
       c & r\\
       y & p
	\end{bmatrix} (\mathtt{mod}\;26) = \begin{bmatrix}
       2 & 17\\
       24 & 15
	\end{bmatrix} (\mathtt{mod}\;26)$$
$$m^{-1} =\;?$$

The plaintext message $m$ is not invertible. We skip the plaintext {\tt\bfseries cr} and use the next pair from the plaintext. Now we have {\tt\bfseries yp} and {\tt\bfseries to}.

$$m = \begin{bmatrix}
       	y & p\\
     	t & o
	\end{bmatrix} (\mathtt{mod}\;26) = \begin{bmatrix}
		24 & 15\\
		19 & 14
	\end{bmatrix} (\mathtt{mod}\;26)$$
$$m^{-1} = \begin{bmatrix}
		12 & 15\\
		19 & 2
	\end{bmatrix} (\mathtt{mod}\;26)$$

The message $m$ is invertible. Now onto finding the key. First we define $c$, the ciphertext that $m$ encrypts to.
$$c = \begin{bmatrix}
		3 & 10\\
		3 & 7
	\end{bmatrix} (\mathtt{mod}\;26)$$
	
Now we can derive the key.
$$c = mK (\mathtt{mod}\;26)$$
$$K = m^{-1}c (\mathtt{mod}\;26) = \begin{bmatrix}
		12 & 15\\
		19 & 2
	\end{bmatrix} \begin{bmatrix}
		3 & 10\\
		3 & 7
	\end{bmatrix} (\mathtt{mod}\;26) = \uuline{\begin{bmatrix}
		3 & 17\\
		11 & 22
	\end{bmatrix}}$$


\subsection*{1.1.2}
With the key $K$ obtained, we now wish to decrypt the text {\tt\bfseries HFFP} mapping to values {\tt\bfseries [7,5,5,15]}.
$$m = cK^{-1} = \begin{bmatrix}
		7 & 5\\
		5 & 15
	\end{bmatrix} \begin{bmatrix}
		3 & 17\\
		11 & 22
	\end{bmatrix}^{-1} (\mathtt{mod}\;26) = \begin{bmatrix}
		7 & 5\\
		5 & 15
	\end{bmatrix} \begin{bmatrix}
		14 & 1\\
		19 & 9
	\end{bmatrix} (\mathtt{mod}\;26) = \begin{bmatrix}
		11 & 0\\
		17 & 10
	\end{bmatrix}$$

The values {\tt\bfseries [11,0,17,10]} maps to the text \uuline{{\tt\bfseries lark}}.

\section*{Exercise 1.2}
\subsection*{1.2.1}
We are to consider $m \times m$ matrices with entries from $\mathbb{Z}_{26}$. We know that the following holds:
$$\det(AB)=\det(A) \cdot \det(B)$$


We have an invertible $m \times m$ matrix $A$ over $\mathbb{Z}_{26}$, where $A = A^{-1}$. We wish to prove:
$$\det(A) = \pm 1\;(\mathtt{mod}\;26)$$

We know that $A$ is invertible, hence we can claim that $A$ is orthogonal.
$$A^{-1}=A^T$$

We know the definition of for an inverse matrix, using identity matrix $I$.
$$A \cdot A^{-1}\;(\mathtt{mod}\;26) = I$$

We can now substitute $A^{-1}$ with $A^T$ and reduce.
$$A \cdot A^T\;(\mathtt{mod}\;26)= I$$
$$\det(A \cdot A^T\;(\mathtt{mod}\;26)) = \det(I) = 1$$
$$\det(A) \cdot \det(A^T)\;(\mathtt{mod}\;26) = 1$$

We know that the determinant of a square matrix and it's transpose are equal.
$$\det(A) \cdot \det(A)\;(\mathtt{mod}\;26) = (\det(A))^2\;(\mathtt{mod}\;26) = 1$$

We can now conclude our proof:
$$\det(A) = \pm 1\;(\mathtt{mod}\;26)$$

\subsection*{1.2.2}

\section*{Exercise 1.3}
\subsection*{1.3.1}

\subsection*{1.3.2}

\end{document}
