\documentclass[11pt]{report}
\usepackage{geometry}
\usepackage{graphicx}
\usepackage{titling}
\usepackage{fancyhdr}
\usepackage{enumerate}
\usepackage{ulem}
\usepackage{latexsym}
\usepackage{float}
\usepackage{amssymb}
\usepackage{amsthm}
\usepackage{amsfonts}
\usepackage{amsmath}
\usepackage[labelfont=bf]{caption}
\usepackage[usenames,dvipsnames,svgnames,table]{xcolor}
\usepackage{listings}
\usepackage{lmodern}
\parindent=0pt
%\frenchspacing

\pagestyle{fancy}

\fancyhead[L]{\slshape\footnotesize May 5, 2014\\\textsc{01410 Cryptology 1}}
\fancyhead[R]{\slshape\footnotesize \textsc{Andreas Kjeldsen (s092638)}\\\textsc{Morten Eskesen (s133304)}}
\fancyfoot[C]{\thepage}
\newcommand{\HRule}{\rule{\linewidth}{0.5mm}}

\definecolor{lightgrey}{rgb}{0.975,0.975,0.975}
\lstdefinestyle{maple}{
	morekeywords={and,assuming,break,by,catch,description,do,done,
	elif,else,end,error,export,fi,finally,for,from,global,if,
	implies,in,intersect,local,minus,mod,module,next,not,od,
	option,options,or,proc,quit,read,return,save,stop,subset,then,
	to,try,union,use,uses,while,xor},
	sensitive=true,
	morecomment=[l]\#,
	morestring=[b]",
	morestring=[d]",
	frame=single,
 	basicstyle=\tt\small,
	keywordstyle=\bf,
	xleftmargin=.25in,
	xrightmargin=.25in,
	mathescape=true,
	backgroundcolor=\color{lightgrey},
}[keywords,comments,strings]

\begin{document}

\begin{titlepage}
\begin{center}

\includegraphics[scale=2.0]{../GFX/dtu_logo.pdf}\\[1cm]
\textsc{\LARGE Technical University of Denmark}\\[1.5cm]
\textsc{\Large 01410 Cryptology 1}\\[0.5cm]

% Title
\HRule \\[0.4cm]
{\huge \bfseries Homework 3}\\[0.1cm]
\HRule \\[1.5cm]

% Author and supervisor
{\large
\emph{Authors:} \\[10pt]
Andreas Hallberg \textsc{Kjeldsen}\\
\emph{s092638@student.dtu.dk} \\[10pt]
Morten Chabert \textsc{Eskesen}\\
\emph{s133304@student.dtu.dk}
}
\vfill

% Bottom of the page
{\large May 5, 2014}

\end{center}
\end{titlepage}

\section*{Exercise 3.1}
\subsection*{3.1.1}
We have to show that $n + d^2$ is a square in $\mathbb{Z}$ where $q - p = 2d > 0$ and $n = pq$ and d, p and q are integers.
\begin{center}
$n + d^2 = pq + (\frac{q-p}{2})^2 = pq + \frac{q^2}{4} + \frac{p^2}{4} - \frac{pq}{2} = \frac{q^2}{4} + \frac{p^2}{4}+\frac{pq}{2} = (\frac{p+q}{2})^2$
\end{center}
Now $p + q = 2d + 2p$ so $\frac{p+q}{2} \in \mathbb{Z}$. This show that $n + d^2$ is a square in $\mathbb{Z}$.

\subsection*{3.1.2}
Given two integers $n$ and $d$ where $n$ is the product of two odd primes $p$ and $q$ and $d$ is a small integer defined as in 3.1.1. We have to show how this information can be used to factor $n$. The primes $p$ and $q$ must be close to each other since $d > 0$ is a small integer given as $\frac{q-p}{2}$ and $q > p$. Since
$$n = (\frac{q + p}{2})^2 - d^2$$
We define the integer $u$ as $u = \frac{p+q}{2}$. $u$ can only be slightly larger than $\sqrt{n}$ and $u^2 - n$ is a square in $\mathbb{Z}$. Therefore we can try the following:
\begin{center}
$u = \lceil \sqrt{n}$ $\rceil + k$, ${}$ ${}$ $k =0,1,2,\dots$
\end{center}
We try this until $u$ becomes a square in $\mathbb{Z}$. Then we calculate the two primes $p$ and $q$ by $p = u - d$ and $q = u + d$ since $n = pq$ with $q > p$ and $n = (\frac{q + p}{2})^2 - d^2$.


\subsection*{3.1.3}
We will use the technique from 3.1.2 to factor $n = 551545081$. We find $\sqrt{n} \approx 23484.99\dots$. We therefore begin our technique with $k = 0$ and find that $u = 23485$.
$$d = \sqrt{u^2 - n} = \sqrt{23485^2 - 551545081} = 12$$
We then get that:\\
$p = u - d = 23485 - 12 = 23473$ and\\
$q = u + d = 23485 + 12 = 23497$\\

\section*{Exercise 3.2}
\subsection*{3.2.1}
Let $n$ be a product of two odd, distinct primes $p_1$ and $p_2$. We have to find the maxmium order of an element modulo $n$.\\
Let $r_1$ be a primitive root mod $p_1$ and let $r_2$ be a primitive root mod $p_2$ and let $t$ = lcm($p_1 -1$, $p_2 - 1$). We use the Chinese Remainder Theorem to find an $x$ such that:
\begin{center}
$x \equiv r_1$ mod $p_1$\\
$x \equiv r_2$ mod $p_2$
\end{center}
We have to show the following:
\begin{enumerate}
\item $x^t \equiv 1$ mod $p_1p_2$.
\item If $0 < k < t$, then $x^k \not\equiv 1$ mod $p_1p_2$.
\item If y $\in$ ($\mathbb{Z}_{p_1p_2}\mathbb{Z}$)$^*$, then $y^t \equiv 1$ mod $p_1p_2$.
\end{enumerate}


\subsection*{3.2.2}
Let $n = 2051152801041163$ (product of two primes) and define the hash function
\begin{center}
$H_F(m) = 8^m$ mod $n$
\end{center}
for $m \in \mathbb{Z}$. The order of 8 modulo $n$ is the maximum possible.\\
Let $p = 2189284635404723$ which is a prime and $\frac{p-1}{2}$ is also a prime.\\
We have to find a primitive element $\alpha \in \mathbb{Z}^*_p$ and choose a valid, private key $a \in \mathbb{Z}_{p-1}$.
The only prime factors in $|\mathbb{Z}^*_p| = p - 1$ are 2 and $\frac{p-1}{2}$ since $\frac{p-1}{2}$ is prime. The order of an element must divide $p - 1$ therefore there are only 4 possible orders: 1 (the identity), 2 (the element - 1), $\frac{p-1}{2}$ and $p - 1$ (the primitive elements).\\
Any element $\alpha \not\equiv \pm 1$ mod $p$ such that $\alpha^{\frac{p-1}{2}} \not\equiv 1$ mod $p$ must have the order $p - 1$ and therefore it is primitive.\\
We use $\alpha = 42$ and use the following command in Maple.
\begin{lstlisting}[style=Maple]
p:=2189284635404723: 42&^((p-1)/2) mod p;
\end{lstlisting}
This shows us that $42^{\frac{p-1}{2}} \not\equiv 1$ mod $p$. It is also clear that $42^2 \not\equiv 1$ mod $p$. Therefore 42 has order $p - 1$ and it is primitive in $\mathbb{Z}^*_p$.\\
We choose the private key $a \in \mathbb{Z}_{p-1}$ at random. Which gives us $a = 815782344718261$.

\subsection*{3.2.3}
We have to use $\alpha, a$ and $p$ to set up the El-Gamal digital signature system. $m$ is an integer describing a 6-digit DTU student number where the leading 0 is discarded if there is any. We compute the signature of $m$ using the El-Gamal system with the hash function $H_F$ and the "random" number $k = 1234567$.\\
We use Morten's student number (133304) as the message, $m = 133304$. We hash $m$ with $H_F$ and we get the following:
\begin{center}
$H_F(133304) \equiv 8^{133304}$\\
$H_F(133304) \equiv 1327930088214640$ mod $2051152801041163$
\end{center}
Now we have our value of $x$. The signature ($\gamma$, $\delta$) $\in \mathbb{Z}_{p} \times \mathbb{Z}_{p-1}$ is then given as:
\begin{center}
$\gamma \equiv 42^{1234567}$\\
$\gamma \equiv 2076571105570857$ mod $2189284635404723$\\
$\delta \equiv (1327930088214640 - 815782344718261 * 2076571105570857)(1234567)^{-1}$\\
$\delta \equiv -1694030045476783892477149105037 * 427810349476471$\\
$\delta \equiv 1297737808822113$ mod $2189284635404722$
\end{center}
Therefore ($\gamma$, $\delta$) = (2076571105570857, 1297737808822113). We found the multiplicative inverse of 1234567 in $\mathbb{Z}^*_{p-1}$ by using the following command in Maple:
\begin{lstlisting}[style=Maple]
1234567^(-1) mod 2189284635404722
\end{lstlisting}
One could also use Euclid's extended algorithm.

\subsection*{3.2.4}
We have to show that the signature produced in 3.2.3 will be verified as the signature on $m$. In order to check that ($\gamma$, $\delta$) = (2076571105570857, 1297737808822113) is verified as the signature on $m$ the following must hold.
\begin{center}
$(\alpha^a)^\gamma \gamma^\delta \equiv \alpha^x$ mod $p$
\end{center}

We start by computing the left-hand side individually.
\begin{center}
$(\alpha^a)^\gamma \equiv (42^{815782344718261})^{2076571105570857} \equiv 1330881686950231$ mod $2189284635404723$\\
$\gamma^\delta \equiv 2076571105570857^{1297737808822113} \equiv 1584897462290462$ mod $2189284635404723$
\end{center}
$(\alpha^a)^\gamma \gamma^\delta \equiv 571999655777925$ mod $2189284635404723$
\\
The right-hand side is
\begin{center}
$\alpha^x \equiv 42^{1327930088214640} \equiv 571999655777925$ mod $2189284635404723$
\end{center}
Therefore $(\alpha^a)^\gamma \gamma^\delta \equiv \alpha^x$ mod $p$ is fulfilled, thus showing that ($\gamma$, $\delta$) = (2076571105570857, 1297737808822113) is verified as the signature on $m$.

\end{document}
