\documentclass[12pt]{report}
\usepackage{graphicx}
\usepackage{titling}
\usepackage{fancyhdr}
\usepackage[latin1]{inputenc}
\usepackage{enumerate}
\usepackage{float}
\usepackage{latexsym}
\usepackage{amssymb}
\usepackage{amsthm}
\usepackage{amsfonts}
\usepackage[usenames,dvipsnames,svgnames,table]{xcolor}
\usepackage{listings}
\parindent=0pt
\frenchspacing

\pagestyle{fancy}

\fancyhead[L]{\slshape\footnotesize November 4, 2014\\\textsc{02515 Health Care Technology}}
\fancyhead[R]{\slshape\footnotesize \textsc{Andreas Kjeldsen (s092638)}\\\textsc{Morten Eskesen (s133304)}}
\fancyfoot[C]{\thepage}

\newcommand{\tab}{\hspace*{2em}}
\newcommand{\HRule}{\rule{\linewidth}{0.5mm}}

\begin{document}

\begin{titlepage}
\begin{center}

\includegraphics[scale=2.0]{../GFX/dtu_logo.pdf}\\[1cm]

\textsc{\LARGE Technical University of Denmark}\\[1.5cm]

\textsc{\Large 02515 Health Care Technology}\\[0.5cm]


% Title
\HRule \\[0.4cm]
{\huge \bfseries 1st Group Report}\\[0.1cm]
\HRule \\[1.5cm]

% Author and supervisor
\large
\emph{Authors:}
\\[10pt]
Andreas Hallberg \textsc{Kjeldsen}\\
\emph{s092638@student.dtu.dk}
\\[10pt]
Morten Chabert \textsc{Eskesen}\\
\emph{s133304@student.dtu.dk}

\vfill

% Bottom of the page
{\large November 4, 2014}

\end{center}
\end{titlepage}

\section*{Progress}
\subsection*{Things that are going well}
The brainstorm has been going good and deciding on an idea that was within our competence and time frame to implement.

\subsection*{Things that are not going so well}
We have no person with a health care background in our group. We would like to do a fitness project that helps people do the exercises properly, but this is hard to do since there are no specific rules for the exercises in the sense that the knees for example should be at a certain angle during squat.

\subsection*{Any changes in the project}
We may change the requirement and test specification during the project since it is quite hard to know what to test at this early stage.

\subsection*{Brief summary of what has been done since last report}
Nothing since this is the first report.

\section*{Description of application}
The application is a game where the person playing has to avoid certain obstacles. The obstacles will come at the person playing where the person playing has to jump or duck to avoid hitting these obstacles. Hitting an obstacle would end the game. The speed of obstacles to avoid would become faster and faster during the game.

\section*{Benefits of application}
The application has an exercise purpose. The squatting down and jumping would cause the person playing to have a higher pulse as these are hard exercises.

\section*{Short litterature study}
"The evaluation of physical activity recommendations: how much is enough", May 2004, Steven N Blair,
Michael J LaMonte, and Milton Z Nichaman\\
${}$\\
"Comparison of loaded and unloaded jump squat training on strength/power performance in college football players", November 2005, Journal of Strength \& Conditioning Research.

\section*{Requirements specification}
\begin{center}
\begin{tabular}{| p {5cm} | p {3.5cm} | p {3.5cm} |}
\hline
Requirement & Need & Nice \\ \hline
Start \& setup time & 1 min & 30 seconds \\ \hline
Performance measurement & Report perfomance & Adapt level of difficulty to performance \\ \hline
Room requirements & Room with no furnitures & Does not matter with furniture \\ \hline
Recognize jumping & Recognize a jump & Recognize the jump's height \\ \hline
Recognize ducking & Recognize a squat & Recognize the player laying on floor \\ \hline
\end{tabular}
\end{center}

\section*{Test specification}
\begin{center}
\begin{tabular}{| p {5cm} | p {3.5cm} | p {3.5cm} |}
\hline
Test & Need & Nice \\ \hline
Start \& setup time & 1 min & 30 seconds \\ \hline
Pulse & 20\% higher pulse than by start of game & 35\% higher pulse \\ \hline
Performance & Performance is reported & Difficulty of performance is adapted to the performance \\ \hline
Jumping & Jumping is recognized & Height of a jump is recognized \\ \hline
User interface & User is instructed and can play & User can play without instructions \\ \hline
\end{tabular}
\end{center}

\section*{Split of workload}
We are both studying computer science and have therefore experience with programming. Therefore we will both work on the implementation of the game. The data analysis and the theoretical parts are also divided 50/50 between us.

\section*{Ideas on what group the applications should be tested on}
We were thinking on testing this game on groups of kids in elementary school. As they will have more energy to play an exercise game for a longer period of time. Meanwhile the setup and user interface of the game will be tested on people between 20-30 as they will have more patience with getting the game started.

\end{document}
