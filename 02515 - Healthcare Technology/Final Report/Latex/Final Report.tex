\documentclass[12pt]{report}
\usepackage{graphicx}
\usepackage{titling}
\usepackage{fancyhdr}
\usepackage[latin1]{inputenc}
\usepackage{enumerate}
\usepackage{float}
\usepackage{latexsym}
\usepackage{amssymb}
\usepackage{amsthm}
\usepackage{amsfonts}
\usepackage[usenames,dvipsnames,svgnames,table]{xcolor}
\usepackage{listings}
\usepackage[hidelinks]{hyperref}
\parindent=0pt
\frenchspacing

\pagestyle{fancy}

\fancyhead[L]{\slshape\footnotesize December 3, 2014\\\textsc{02515 Health Care Technology}}
\fancyhead[R]{\slshape\footnotesize \textsc{Andreas Kjeldsen (s092638)}\\\textsc{Morten Eskesen (s133304)}}
\fancyfoot[C]{\thepage}

\newcommand{\tab}{\hspace*{2em}}
\newcommand{\HRule}{\rule{\linewidth}{0.5mm}}

\begin{document}

\begin{titlepage}
\begin{center}

\includegraphics[scale=2.0]{../GFX/dtu_logo.pdf}\\[1cm]

\textsc{\LARGE Technical University of Denmark}\\[1.5cm]

\textsc{\Large 02515 Health Care Technology}\\[0.5cm]


% Title
\HRule \\[0.4cm]
{\huge \bfseries Training Memory Game}\\[0.1cm]
\HRule \\[1.5cm]

% Author and supervisor
\large
\emph{Authors:}
\\[10pt]
Andreas Hallberg \textsc{Kjeldsen}\\
\emph{s092638@student.dtu.dk}
\\[10pt]
Morten Chabert \textsc{Eskesen}\\
\emph{s133304@student.dtu.dk}

\vfill

% Bottom of the page
{\large December 3, 2014}

\end{center}
\end{titlepage}

\chapter{Introduction}
\section{Formalities}
This report is a result of a project done in the course \emph{Health Care Technology} taken at DTU. The project has been a collaboration between us both, Andreas and Morten, where we each have been equally responsible for all parts of the project. By Andreas' mother we were able to test the game on kids from elementary school. 

\section{Objectives}
The objective and product of the course Health Care Technology is to create a health-promoting interactive game by using Unity and Kinect technology. The game should solve a health care problem in Denmark. The course puts a lot of focus on health care challenges in today's Denmark and how the challenges can be solved by modern technology. In the future we will rely more and more on modern technology for health care problem as one of the problems in health care is the increasing need for treatment of the aging population. Another problem is the number of obese people in Denmark which has increased considerably in the last decade \cite{Sundhedsstyrelsen}.\\
This project aims to aid the solving of the increasing obesity problem in Denmark by creating a game that will increase the physical activity of children such that obesity can be prevented and obese children can become more healthy.

\section{Obesity}


\chapter{Game details}
\section{Requirements specification}


\section{Game description}


\section{Technology}


\section{Game specification}


\chapter{Implementation}


\section{Implementation order}


\chapter{Test}
\section{Test specification}


\section{Test results}


\chapter{Conclusion \& Discussion}


\chapter{Further development}

\begin{thebibliography}{9}
\bibitem{Sundhedsstyrelsen}
  Sundhedsstyrelsen:
  \emph{Overv\ae gt}
  \url{http://sundhedsstyrelsen.dk/da/sundhed/overvaegt}

\end{thebibliography}


\chapter{Appendixes}

\end{document}
