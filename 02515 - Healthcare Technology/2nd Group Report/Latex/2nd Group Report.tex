\documentclass[12pt]{report}
\usepackage{graphicx}
\usepackage{titling}
\usepackage{fancyhdr}
\usepackage[latin1]{inputenc}
\usepackage{enumerate}
\usepackage{float}
\usepackage{latexsym}
\usepackage{amssymb}
\usepackage{amsthm}
\usepackage{amsfonts}
\usepackage[usenames,dvipsnames,svgnames,table]{xcolor}
\usepackage{listings}
\parindent=0pt
\frenchspacing

\pagestyle{fancy}

\fancyhead[L]{\slshape\footnotesize November 18, 2014\\\textsc{02515 Health Care Technology}}
\fancyhead[R]{\slshape\footnotesize \textsc{Andreas Kjeldsen (s092638)}\\\textsc{Morten Eskesen (s133304)}}
\fancyfoot[C]{\thepage}

\newcommand{\tab}{\hspace*{2em}}
\newcommand{\HRule}{\rule{\linewidth}{0.5mm}}

\begin{document}

\begin{titlepage}
\begin{center}

\includegraphics[scale=2.0]{../GFX/dtu_logo.pdf}\\[1cm]

\textsc{\LARGE Technical University of Denmark}\\[1.5cm]

\textsc{\Large 02515 Health Care Technology}\\[0.5cm]


% Title
\HRule \\[0.4cm]
{\huge \bfseries 2nd Group Report}\\[0.1cm]
\HRule \\[1.5cm]

% Author and supervisor
\large
\emph{Authors:}
\\[10pt]
Andreas Hallberg \textsc{Kjeldsen}\\
\emph{s092638@student.dtu.dk}
\\[10pt]
Morten Chabert \textsc{Eskesen}\\
\emph{s133304@student.dtu.dk}

\vfill

% Bottom of the page
{\large November 18, 2014}

\end{center}
\end{titlepage}

\section*{Progress}
\subsection*{Things that are going well}
Working with the Microsoft Kinect has become a bit easier. We have a better understanding of how it works, what data it can provide and how we can use that data. Gesture recognition is quite fun to implement and test.\\
We have working prototype of our game.

\subsection*{Things that are not going so well}
We found out that the game we originally planned to make, was not doable the way we wanted to do it. Therefore we have changed decided to change the game a bit. Primarily we had major difficulties with building a proper scene, this includes finding good and assets to make it look nice and also finding an avatar that would actually work with the Microsoft Kinect. We tried different avatars, but there always seemed to be something wrong.

\subsection*{Any changes in the project}
We agreed to forget about making proper graphics and using avatars. Instead we've chosen to use the RGB Camera feature of the Microsoft Kinect to display the actual user. Then we've made an overlay on top of the RGB image displaying the game components, such as round, score, moves to make.\\
The game has been changed to a Memory Game of training exercises. Round 1 has 1 exercise, round 2 has 2 exercises and so forth. When a round begins the exercises to perform are displayed and then a timer is started. The player has to perform the displayed exercises within a timeframe. The exercises are only displayed in the beginning of the round, which means that not only do they have to perform the exercise, they also have to remember the order in which to perform them. Right now we support to exercises, \emph{Jump} and \emph{Squat}. The exercises are chosen at random. When an exercise is chosen, it gets added to the list of exercises to perform, which means the exercises already in the list remain there when a new round starts and a new exercise is added.

\subsection*{Brief summary of what has been done since last report}
The new game has been prototyped and tested briefly. The requirement specification is about done, the test specification might be extended. Most tests pass and some are yet to be executed. We expect to be able to improve the prototype a bit before we get our chosen test groups to try it out.

\end{document}
